\item {[}ALVL/2017/P1/11{]}

Sir Isaac Newton was a famous scientist renowned for his work on the
laws of motion. One law states that. for an object falling vertically
in a vacuum. the rate of change of velocity, $v$ $\mathrm{ms^{-1}}$,
with respect to time, $t$ seconds, is a constant, $c$.
\begin{enumerate}
\item {}
\begin{enumerate}
\item Write down a differential equation relating $v$, $t$ and $c$. \hfill{}{[}1{]}
\item Initially the velocity of the object is 4 $\mathrm{ms^{-1}}$ and,
after a further 2.5 s. the velocity of the object is $29$ $\mathrm{ms^{-1}}$.
Find $v$ in terms of $t$ and state the value of $c$. \hfill{}{[}3{]}
\end{enumerate}
\end{enumerate}
For an object falling vertically through the atmosphere. the rate
of change of velocity is less than that for an object falling in a
vacuum. The new rate of change of $v$ is modelled as the difference
between the value of $c$ found in part (i)(b) and an amount proportional
to the velocity $v$, with a constant of proportionality $k$.
\begin{enumerate}
\item Given that in this case the initial velocity is zero, find $v$ in
terms of $t$ and $k$. \hfill{}{[}5{]}
\end{enumerate}
For an object falling through the atmosphere, the \textquoteleft terminal
velocity\textquoteleft{} is the value approached by the velocity after
a long time.
\begin{enumerate}
\item A falling object has initial velocity zero and terminal velocity $40$
$\mathrm{ms^{-1}}$. Find how long it takes the object to reach 90\%
of its terminal velocity.\hfill{} {[}4{]}
\end{enumerate}
