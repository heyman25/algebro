\item {[}ALVL/2017/P1/9{]}
\begin{enumerate}
\item A sequence of numbers $u_{1},u_{2},u_{3},\dots$ has a sum $S_{n}$
where $S_{n}=\sum_{r=1}^{n}u_{r}$. It is given that $S_{n}=An^{2}+Bn$,
where $A$ and $B$ are non-zero constants. 
\begin{enumerate}
\item Find an expression for $u_{n}$ in terms of $A$, $B$ and $n$. Simplify
your answer. \hfill{}{[}3{]}
\item It is also given that the tenth term is 48 and the seventeenth term
is 90. Find $A$ and $B$. \hfill{}{[}2{]}
\end{enumerate}
\item Show that $r^{2}\left(r+1\right)^{2}-\left(r-1\right)^{2}r^{2}=kr^{3}$,
where $k$ is a constant to be determined. Use this result to find
a simplified expression for $\sum_{r=1}^{n}r^{3}$. \hfill{}{[}4{]}
\item D'Alembert's ratio test states that a series of the form $\sum_{r=0}^{\infty}a_{r}$,
converges when $\lim_{n\rightarrow\infty}\left|\frac{a_{n+1}}{a_{n}}\right|<1$,
and diverges when $\lim_{n\rightarrow\infty}\left|\frac{a_{n+1}}{a_{n}}\right|>1$.
When $\lim_{n\rightarrow\infty}\left|\frac{a_{n+1}}{a_{n}}\right|=1$,
the test is inconclusive. Using the test, explain why the series $\sum_{r=0}^{\infty}\frac{x^{r}}{r!}$
converges for all real values of $x$ and state the sum to infinity
of this series, in terms of $x$.\hfill{} {[}4{]}
\end{enumerate}
