\item {[}ALVL/2017/P2/9{]}

On average 8\% of a certain brand of kitchen lights are faulty. The
lights are sold in boxes of 12. 
\begin{enumerate}
\item State, in context, two assumptions needed for the number of faulty
lights in a box to be well modelled by a binomial distribution. \hfill{}{[}2{]}
\end{enumerate}
Assume now that the number of faulty lights in a box has a binomial
distribution.
\begin{enumerate}
\item Find the probability that a box of 12 of these kitchen lights contains
at least 1 faulty light. \hfill{}{[}1{]}
\end{enumerate}
The boxes are packed into cartons. Each carton contains 20 boxes.
\begin{enumerate}
\item Find the probability that each box in one randomly selected carton
contains at least one faulty light.\hfill{} {[}1{]}
\item Find the probability that there are at least 20 faulty lights in a
randomly selected carton. \hfill{}{[}2{]}
\item Explain why the answer to part (iv) is greater than the answer to
part (iii). \hfill{}{[}1{]}
\end{enumerate}
The manufacturer introduces a quick test to check if lights are faulty.
Lights identified as faulty are discarded. If a light is faulty there
is a 95\% chance that the quick test will correctly identify the light
as faulty. If the light is not faulty, there is a 6\% chance that
the quick test will incorrectly identify the light as faulty. 
\begin{enumerate}
\item Find the probability that a light identified as faulty by the quick
test is not faulty.\hfill{} {[}3{]}
\item Find the probability that the quick test correctly identifies lights
as faulty or not faulty. \hfill{}{[}1{]}
\item Discuss briefly whether the quick test is worthwhile.\hfill{} {[}1{]}
\end{enumerate}
