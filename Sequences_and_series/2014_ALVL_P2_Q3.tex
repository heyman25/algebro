\item {[}ALVL/2014/P2/3{]}

In a training exercise, athletes run from a starting point $O$ to
and from a series of points, $A_{1}$, $A_{2}$, $A_{3}$, $\dots$,
increasingly far away in a straight line. In the exercise, athletes
start at $O$ and run stage 1 from $O$ to $A_{1}$ and back to $O$,
then stage 2 from $O$ to $A_{2}$ and back to $O$, and so on. 
\begin{enumerate}
\item {}
\noindent \begin{center}
<INSERT DIAGRAM HERE>
\par\end{center}

In Version 1 of the exercise, the distances between adjacent points
are all 4 m (see Fig. l). 
\begin{enumerate}
\item Find the distance run by an athlete who completes the first 10 stages
of Version 1 of the exercise.\hfill{} {[}2{]}
\item Write down an expression for the distance run by an athlete who completes
$n$ stages of Version 1. Hence find the least number of stages that
the athlete needs to complete to run at least 5 km.\hfill{} {[}4{]}
\end{enumerate}
\item {}
\noindent \begin{center}
<INSERT DIAGRAM HERE>
\par\end{center}

In Version 2 of the exercise. the distances between the points are
such that $OA_{1}=4\text{m}$, $A_{1}A_{2}=4\text{m}$, $A_{2}A_{3}=8\text{m}$
and $A_{n}A_{n+1}=2A_{n-1}A_{n}$ (see Fig. 2). Write down an expression
for the distance run by an athlete who completes $n$ stages of Version
2. Hence find the distance from $O$, and the direction of travel,
of the athlete after he has run exactly 10 km using Version 2.\hfill{}
{[}5{]}
\end{enumerate}
