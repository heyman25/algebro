\item {[}ALVL/2012/P2/4{]}

On 1 January 2001 Mrs $A$ put \$100 into a bank account, and on the
first day of each subsequent month she put in \$10 more than in the
previous month. Thus on 1 February she put \$110 into the account
and on 1 March she put \$120 into the account, and so on. The account
pays no interest.
\begin{enumerate}
\item On what date did the value of Mrs A's account first become greater
than \$5000? {[}5{]} 
\end{enumerate}
On 1 January 2001 Mr B put \$100 into a savings account, and on the
first day of each subsequent month he put another \$100 into the account.
The interest rate was 0.5\% per month, so that on the last day of
each month the amount in the account on that day was increased by
0.5\%. 
\begin{enumerate}
\item Use the formula for the sum of a geometric progression to find an
expression for the value of Mr B\textquoteleft s account on the last
day of the $n$th month (where January 2001 was the lst month. February
2001 was the 2nd month, and so on). Hence find in which month the
value of Mr B\textquoteleft s account first became greater than \$5000.
{[}5{]}
\item Mr $B$ Wanted the value of his account to be \$5000 on 2 December
2003. What interest rate per month, applied from January 2001 , would
achieve this? {[}3{]}
\end{enumerate}
