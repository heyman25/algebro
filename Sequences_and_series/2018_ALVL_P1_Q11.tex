\item {[}ALVL/2018/P1/11{]}

Mr Wong is considering investing money in a savings plan. One plan,
$P$, allows him to invest \$100 into the account on the first day
of every month. At the end of each month the total in the account
is increased by $a%\%
$. 
\begin{enumerate}
\item It is given that $a=0.2$.
\begin{enumerate}
\item Mr Wong invests \$100 on 1 January 2016. Write down how tnuch this
\$100 is worth at the end of 31 December 2016. \hfill{}{[}1{]}
\item Mr Wong invests \$100 on the first day of each of the 12 months of
2016. Find the total amount in the account at the end of 31 December
2016. \hfill{}{[}3{]}
\item Mr Wong continues to invest \$100 on the first day of each month.
Find the month in which the total in the account will first exceed
\$3000. Explain whether this occurs on the first or last day of the
month. \hfill{}{[}5{]}
\end{enumerate}
\end{enumerate}
An alternative plan. Q. also allows him to invest \$100 on the first
day of every month. Each \$100 invested earns a fixed bonus of \$$b$
at the end of every month for which it has been in the account. This
bonus is added to the account. The accumulated bonuses themselves
do not earn any further bonus. 
\begin{enumerate}
\item {}
\begin{enumerate}
\item Find, in terms of $b$, how much \$100 invested on 1 January 2016
will be worth at the end of 31 December 2016. \hfill{}{[}1{]}
\item Mr Wong invests \$100 on the first day of each of the 24 months in
2016 and 2017. Find the value of $b$ such that the total value of
all the investments, including bonuses, is worth \$2800 at the end
of 31 December 2017. \hfill{}{[}3{]}
\end{enumerate}
\end{enumerate}
It is given instead that $a=1$ for plan $P$.
\begin{enumerate}
\item Find the value of $b$ for plan $Q$ such that both plans give the
same total value in the account at the end of the 60th month.\hfill{}
{[}3{]}
\end{enumerate}
