\item {[}ALVL/2013/P2/12{]}

A company has two departments and each department records the number
of employees absent through illness each day. Over a long period of
time it is found that the average numbers absent on a day are 1.2
for the Administration Department and 2.7 for the Manufacturing Department.
\begin{enumerate}
\item State, in this context, two conditions that must be met for the numbers
of absences to be well modelled by Poisson distributions. Explain
why each of your two conditions may not be met. \hfill{}{[}3{]}
\end{enumerate}
For the remainder of this question assume that these conditions are
met. You should assume also that absences in the two departments are
independent of each other.
\begin{enumerate}
\item Find the smallest number of days for which the probability that no
employee is absent througzh illness from the Administration Department
is less than 0.01. \hfill{}{[}2{]}
\end{enumerate}
Each employee absent on a day represents one \textquoteleft day of
absence\textquoteleft . So, one employee absent for 3 days contributes
3 days of absence, and 5 employees absent on 1 day contribute 5 days
of absence.

(iii) Find the probability that, in a 5-day period, the total number
of days of absence in the tug; departments is more than 20. \hfill{}
{[}3{]}

(iv) Use a suitable approximation, which should be stated together
with its parameter(s), to find the probability that, in a 60-day period.
the total number of days of absence tn the two department is between
200 and 250 inclusive. \hfill{}{[}4{]}
