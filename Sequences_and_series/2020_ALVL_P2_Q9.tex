\item {[}ALVL/2020/P2/9{]}

A factory produces ballpoint pens. On average 6\% of the pens are
faulty. The pens are packed in boxes of 100 for sale to retail outlets.
It should be assumed that the number of faulty pens in a box of 100
pens follows a binomial distribution. 

For quality control purposes a random sample of 10 pens from each
box is tested. If 2 or fewer faulty pens are found in the sample of
10, the box is accepted for sale. Otherwise the box is rejected. 
\begin{enumerate}
\item Explain what is meant by a random sample in this contexts. \hfill{}{[}1{]}
\item Find the probability that a randomly chosen box of 100 pens is accepted
for sale. \hfill{}{[}1{]}
\item One morning 75 boxes are tested in this way. Find the probability
that more than 5\% of these boxes are rejected. \hfill{}{[}4{]}
\end{enumerate}
An alternative testing procedure is trialled in which a random sample
of 5 pens is initially taken from a box and tested.
\begin{itemize}
\item If there are no faulty peas in this sample of 5 the box is accepted.
\item If there are 3 or more faulty pens in this sample of 5 the box is
rejected.
\item If there are l or 2 faulty pens in this sample a second random sample
of 5 pens is taken from the box. When the second sample has been tested.
the box is accepted if the total number of faulty pens found in the
combined sample of ID is 2 or fewer and rejected otherwise.
\end{itemize}
\begin{enumerate}
\item Find the probability that a randomly chosen box of 100 pens is accepted
for sale when the alternative testing procedure is used. \hfill{}{[}5{]}
\item Explain why the factory manager might prefer to use the alternative
testing procedure. \hfill{}{[}1{]}
\end{enumerate}
