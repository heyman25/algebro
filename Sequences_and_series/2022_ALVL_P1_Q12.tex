\item {[}ALVL/2022/P1/12{]}

Scientists are interested in the population of a particular species.
They attempt to model the population $P$ at time $t$ days using
a differential equation. Initially the population is observed to be
50 and afier 10 days the population is 100.

The first model the scientists use assumes that the rate of change
of the population is proportional to the population.
\begin{enumerate}
\item Write down a differential equation for this model and solve it for
$P$ in terms of $t$.\hfill{} {[}5{]} 
\end{enumerate}
To allow for constraints on population growth, the model is refined
to 

\[
\frac{dP}{dt}=\lambda P\left(500-P\right)
\]

where $\lambda$ is a constant. 
\begin{enumerate}
\item[(b)]  Solve this differential equation to find $P$ in terms of $t$.\hfill{}
{[}6{]}
\item[(c)]  Using the refined model. state the population of this species in
the long term. Comment on how this value suggests the refined model
is an improvement on the first model.\hfill{} {[}2{]}
\end{enumerate}
