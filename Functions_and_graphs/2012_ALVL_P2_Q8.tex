\item {[}ALVL/2012/P2/8{]}

Amy is revising fora mathematics examination and takes a different
practice paper each week. Her marks, $y\%$ in week $x$, are as follows.
\noindent \begin{center}
\begin{tabular}{|c|c|c|c|c|c|c|}
\hline 
Week $x$ & 1 & 2 & 3 & 4 & 5 & 6\tabularnewline
\hline 
Percentage mark $y$ & 38 & 63 & 67 & 75 & 71 & 82\tabularnewline
\hline 
\end{tabular}
\par\end{center}
\begin{enumerate}
\item Draw a scatter diagram showing these marks. {[}1{]}
\item Suggest a possible reason why one of the marks does not seem to follow
the trend. {[}1{]}
\item It is desired to predict Amy's marks on future papers. Explain why.
in this context. neither a linear nor a quadratic model is likely
to be appropriate. {[}2{]} 
\end{enumerate}
It is decided to fit a model of the form $\ln\left(L-y\right)=a+bx$,
where $L$ is a suitable constant. The product moment correlation
coefiicient between $x$ and $\ln\left(L-y\right)$is denoted by $r$.
The following table gives values of $r$ for some possible values
of $L$.
\noindent \begin{center}
\begin{tabular}{|c|c|c|c|}
\hline 
$L$ & 91 & 92 & 93\tabularnewline
\hline 
$r$ &  & -0.929 944 & -0.929 918\tabularnewline
\hline 
\end{tabular}
\par\end{center}
\begin{enumerate}
\item Calculate the value of $r$ for $L=91$, giving your answer correct
to 6 decimal places. {[}1{]}
\item Use the table and your answer to part (iv) to suggest with a reason
which of 91, 92 or 93 is the most appropriate value for $L$. {[}1{]}
\item Using this value for $L$, calculate the values of $a$ and $b$,
and use them to predict the week in which Amy will obtain her first
mark of at least 90\%. {[}4{]}
\item Give an interpretation, in context, of the value of L. {[}1{]}
\end{enumerate}
