\item {[}ALVL/2015/P1/11{]}

A curve $C$ has parametric equations 
\[
x=\sin^{3}\theta,\,y=3\sin^{2}\theta\cos\theta,\,\text{for }0\leq\theta\le\frac{1}{2}\pi.
\]
 
\begin{enumerate}
\item Show that $\frac{dy}{dx}=2\cot\theta-\tan\theta$.\hfill{} {[}3{]}
\item Show that $C$ has a turning point when $\tan\theta=\sqrt{k}$, where
$k$ is an integer to be determined. Find, in non-trigonometric form,
the exact coordinates of the turning point and explain why it is a
maximum. \hfill{}{[}6{]} 
\item Show that the area of the region bounded by $C$ and the $x$-axis
is given by 
\[
\int_{0}^{\frac{1}{2}\pi}9\sin^{4}\theta\cos^{2}\theta\,d\theta.
\]
Use your calculator to find the area, giving your answer correct to
3 decimal places. \hfill{}{[}3{]}
\end{enumerate}
The line with equation $y=ax$, where $a$ is a positive constant,
meets $C$ at the origin and at the point $P$. 
\begin{enumerate}
\item Show that $\tan\theta=\frac{3}{a}$ at $P$. Find the exact value
of $a$ such that the line passes through the maximum point of $C$.
\hfill{}{[}3{]}
\end{enumerate}
