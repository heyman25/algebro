\item {[}ALVL/2021/P1/10{]}

Scientists model the number of bacteria. $N$, present at a time $t$
minutes after setting up an experiment. The model assumes that, at
any time $t$, the growth rate in the number of bacteria is $kN$,
for some positive constant $k$. Initially there are 100 bacteria
and it is found that there are 300 bacteria at time $t=2$.
\begin{enumerate}
\item Write down and solve a differential equation involving $N$, $t$
and $k$. Find $k$ and the time it takes for the number of bacteria
to reach 1000. \hfill{}{[}4{]}
\end{enumerate}
The scientists repeat the experiment, again with an initial number
of 100 bacteria. The growth rate, $kN$, for the number of bacteria
is the same as that found in part (a). This time they add an anti-bacterial
solution which they model as reducing the number of bacteria by $d$
bacteria per minute.
\begin{enumerate}
\item[(b)]  Write down and solve a differential equation, giving $t$ in terms
of $N$ and $d$. Hence find $N$ in terms of $t$ and $d$. \hfill{}{[}5{]}
\item {}
\begin{enumerate}
\item Find the range of values of $d$ for which the number of bacteria
will decrease.\hfill{} {[}1{]}
\item In the case where $d=58$, find the time taken for the number of bacteria
to reach zero. \hfill{}{[}2{]}
\end{enumerate}
\end{enumerate}
