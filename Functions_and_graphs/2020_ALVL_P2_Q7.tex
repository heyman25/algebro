\item {[}ALVL/2020/P2/7{]}

A study into the germination of parsnip seeds in the 1980s produced
the following data for the average number of days, $d$, taken for
a seed to gemtinnte at different soil temperatures, $t$, measured
in degrees Fahrenheit.
\noindent \begin{center}
\begin{tabular}{|c|c|c|c|c|c|}
\hline 
$t$ & 32 & 41 & 50 & 59 & 68\tabularnewline
\hline 
$d$ & 172 & 57 & 27 & 19 & 14\tabularnewline
\hline 
\end{tabular}
\par\end{center}
\begin{enumerate}
\item Sketch a scatter diagram of the data. State the product moment correlation
coefficient between $d$ and $t$. \hfill{}{[}2{]}
\item Lim thinks the data can be modelled by the regression equation $d=-a+bu$,
where $u=\frac{1}{t}$. Find the values of $a$ and $b$ for Lim's
model, giving the values correct to 3 significant figures. State the
product moment correlation coefficient between $d$ and $u$. \hfill{}{[}3{]}
\end{enumerate}
The study also found that, at a soil temperature of 86 degrees Fahrenheit.
parsnip seeds took an average of 32 days to germinate.
\begin{enumerate}
\item Determine whether Lim's model fit; this additional data.\hfill{}
{[}1{]}
\end{enumerate}
A temperature of $F$ degrees Fahrenheit is equivalent to a temperature
of $C$ degrees Celsius, where $C=\frac{5}{9}\left(F-32\right)$.
\begin{enumerate}
\item Write Lim\textquoteleft s equation from part (ii) in terms of $d$
and $T$, where $T$ is the temperature in degrees Celsius.\hfill{}
{[}2{]}
\end{enumerate}
