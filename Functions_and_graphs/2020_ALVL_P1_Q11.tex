\item {[}ALVL/2020/P1/11{]}

In sport science. studies are made to optimise performance in all
aspects of sport from fitness to technique.
\noindent \begin{center}
\textless INSERT DIAGRAM HERE\textgreater{}
\par\end{center}

In a game. a player scores 3 points by carrying the ball over the
scoring line, shown in the diagram as $XY$. When a player has scored
these 3 points, an extra point is scored if the ball is kicked between
two fixed vertical posts at C and D. The kick can be taken from any
point on the line $AB$, where $A$ is the point at which the player
crossed the scoring line and $AB$ is perpendicular to $XY$.

The distance $CD$ is 4 m; $XC$ is equal to $DY$; the point $A$
is a distance $a$ m from $C$ and $A$ lies between $X$ and $C$.
The kick is taken from the point $K$, where $AK$ is $x$ m. The
angle $CKD$ is $\theta$ (Sec diagram).
\begin{enumerate}
\item By expressing $\theta$ as the difference of two angles, or otherwise,
show that 
\[
\tan\theta=\frac{4x}{x^{2}+4a+a^{2}}.
\]
\hfill{}{[}1{]} 
\item Find, in terms of $a$, the value of $x$ which maximises $\tan\theta$,
simplifying your answer. Find also the corresponding value of $\tan\theta$.
(You need not show that your answer gives a maximum.) \hfill{}{[}3{]}
\end{enumerate}
The point corresponding to the value of $x$ found in part (ii) is
called the optimal point. The corresponding value of $\theta$ is
called the optimal angle.
\begin{enumerate}
\item Explain why a player may decide not to take the kick from the optimal
point. \hfill{}{[}1{]}
\item Show that, when $\theta$ is the optimal angle, $\tan KDA=\sqrt{\frac{a}{4+a}}$.
Find the approximate value of angle $KDA$ when $a$ is much greater
than 4.\hfill{} {[}3{]}
\item It is given that the length of the scoring line $XY$ is 50 m. Find
the range in which the optimal angle lies as the location of $A$
varies between $X$ and $C$.\hfill{} {[}2{]}
\end{enumerate}
