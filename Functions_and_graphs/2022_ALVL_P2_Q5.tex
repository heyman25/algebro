\item {[}ALVL/2022/P2/5{]}

The diagram shows part ofthe circle $x^{2}+y^{2}=r^{2}$ and the line
$y=r-h$, where $0<h<r$. The shaded region between the circle and
the line is rotated about the $y$-axis to form a solid, which ts
called a spherical cap. The height of the spherical cap is $h$. 
\begin{enumerate}
\item Show by integration that the volume ofthe spherical cap is $\frac{1}{3}\pi h^{2}\left(3r-h\right)$.
\hfill{}{[}5{]}
\noindent \begin{center}
<INSERT DIAGRAM HERE>
\par\end{center}

\end{enumerate}
An ornament is made from a solid sphere of radius 15 cm by removing
two spherical caps, one of height $p$ cm from the top of the sphere
and the other of height $3p$ cm from the bottom of the sphere. The
plane faces of the ornament are parallel (see diagram). The volume
of the ornament, shown shaded, is $3402\pi$ $\mathrm{cm}^{3}$.

{[}It is given that the volume of a sphere of radius $r$ is $\frac{4}{3}\pi r^{3}$
.{]} 
\begin{enumerate}
\item[(b)]  Find the cubic equation satisfied by $p$, and hence find the value
of $p$. \hfill{}{[}4{]}
\end{enumerate}
A different ornament is made by making two parallel cuts to another
sphere ofradius 15 cm.
\begin{itemize}
\item The volume of this second ornament is less than the volume ofthe omnment
in part (b).
\item The top face of this second ornament has the same radius as the top
face of the ornament In part (b). 
\item The bottom face of this second ornament has the same radius as the
bottom face of the ornamem in part (b). 
\end{itemize}
\begin{enumerate}
\item[(c)]  Find the volume ofthis second omament. Give your answer as an exact
multiple of $\pi$.\hfill{} {[}2{]}
\end{enumerate}
