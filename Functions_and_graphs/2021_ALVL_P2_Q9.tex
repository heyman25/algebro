\item {[}ALVL/2021/P2/9{]}

In a chemical reaction the mass, $x$ grams, of a particular product
at time $t$ minutes is given in this table. 
\begin{center}
\begin{tabular}{|c|c|c|c|c|c|c|c|c|c|}
\hline 
$t$ & 0 & 3 & 6 & 9 & 12 & 15 & 18 & 21 & 24\tabularnewline
\hline 
$x$ & 10.1 & 11.3 & 14.3 & 16.7 & 19.5 & 22.8 & 28.7 & 32.5 & 39.3\tabularnewline
\hline 
\end{tabular}
\par\end{center}

The value of the product moment correlation coellicient is 0.9803
correct to 4 significant figures. The scatter diagram for the data
is shown below.
\noindent \begin{center}
<INSERT DIAGRAM HERE>
\par\end{center}
\begin{enumerate}
\item Toby attempts to model the relationship between x and t with a straight
line. Explain whether this is likely to provide a good model. \hfill{}
{[}1{]}
\end{enumerate}
Toby now tries a model in which $x$ has been transformed to $\ln x$.
\begin{enumerate}
\item[(b)]  {} 
\begin{enumerate}
\item Sketch a scatter diagram of $\ln x$ against $t$ for the data given
in the table. \hfill{}{[}1{]}
\item Toby models the data with the equation $\ln x=c+dt$. Find the values
of the constants $c$ and $d$ and state the value of the product
moment correlation coefficient for this model. \hfill{}{[}3{]}
\end{enumerate}
\item[(c)] Comment on Toby\textquoteleft s two models. \hfill{}{[}2{]}
\end{enumerate}
