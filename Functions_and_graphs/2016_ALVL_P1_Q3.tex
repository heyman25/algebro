\item {[}ALVL/2016/P1/3{]}

The curve $y=x^{4}$ is transformed onto the curve with equation $y=f\left(x\right)$.
The turning point on $y=x^{4}$ corresponds to the point with coordinates
$\left(a,b\right)$ on $y=f\left(x\right)$. The curve $y=f\left(x\right)$
also passes through the point with coordinates $\left(0,c\right)$.
Given that $f\left(x\right)$ has the form $k\left(x-l\right)^{4}+m$
and that $a$, $b$ and $c$ are positive constants with $c>b$, express
$k$, $l$ and $m$ in terms of $a$, $b$ and $c$. \hfill{} {[}2{]}

By sketching the curve $y=f\left(x\right)$, or otherwise, sketch
the curve $y=\frac{1}{f\left(x\right)}$. State, in terms of $a$,
$b$ and $c$, the coordinates of any pomts where $y=\frac{1}{f\left(x\right)}$
crosses the axes and of any turning points. \hfill{}{[}4{]}
