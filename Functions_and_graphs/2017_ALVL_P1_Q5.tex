\item {[}ALVL/2017/P1/5{]}

When the polynomial $x^{3}+ax^{:2}+bx+c$ is divided by $\left(x-1\right)$,
$\left(x-2\right)$ and $\left(x-3\right)$, the remainders are 8,
l2 and 25 respectively.
\begin{enumerate}
\item Find the values of $a$, $b$ and $c$. \hfill{}{[}4{]}
\end{enumerate}
A curve has equation $y=f\left(x\right)$. where $f\left(x\right)=x^{3}+ax^{2}+bx+c$,
with the values of $a$, $b$ and $c$ found in part (i).
\begin{enumerate}
\item Show that the gradient of the curve is always positive. Hence explain
why the equation $f\left(x\right)=0$ has only one real root and find
this root.\hfill{} {[}3{]}
\item Find the $x$-coordinates of the points where the tangent to the curve
is parallel to the line $y=2x-3$. \hfill{}{[}3{]}
\end{enumerate}
