\item {[}ALVL/2017/P2/8{]}
\begin{enumerate}
\item Draw separate scatter diagrams, each with 8 points, all in the first
quadrant, which represent the situation where the product moment correlation
coefficient between variables $x$ and $y$ is 
\begin{enumerate}
\item $-1$, 
\item 0, 
\item between 0.5 and 0.9.\hfill{} {[}3{]}
\end{enumerate}
\item An investigation into the effect of a fertiliser on yields of corn
found that the amount of fertiliser applied, $x$, resulted in the
average yields of corn, $y$, given below. where $x$ and $y$ are
measured in suitable units. 
\noindent \begin{center}
\begin{tabular}{|c|c|c|c|c|c|c|}
\hline 
$x$ & 0 & 40 & 80 & 120 & 160 & 200\tabularnewline
\hline 
$y$ & 70 & 104 & 118 & 119 & 126 & 129\tabularnewline
\hline 
\end{tabular}
\par\end{center}
\begin{enumerate}
\item Draw a scatter diagram for these values. State which of the following
equations. where $a$ and $b$ are positive constants, provides the
most accurate model of the relationship between $x$ and $y$. 
\begin{enumerate}
\item $y=ax^{2}+b$
\item $y=\frac{a}{x^{2}}+b$
\item $y=a\ln2x+b$
\item $y=a\sqrt{x}+b$ ,\hfill{} {[}2{]}
\end{enumerate}
\item Using the model you chose in part (i), write down the equation for
the relationship between $x$ and $y$, giving the numerical values
of the coefficients. State the product moment correlation coefficient
for this model. {[}3{]} 
\item Give two reasons why it would be reasonable to use your model to estimate
the value of $y$ when $x=189$. \hfill{}{[}2{]}
\end{enumerate}
\end{enumerate}
