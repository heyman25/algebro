\item {[}ALVL/2019/P1/12{]}
\noindent \begin{center}
\textless INSERT\_DIAGRAM\_HERE\textgreater{}
\par\end{center}

A ray of light passes from air into a material made into a rectangular
prism. The ray of light is sent in direction $\left(\begin{array}{c}
-2\\
-3\\
-6
\end{array}\right)$ from a light source at the point $P$ with coordinates $\left(2,2,4\right)$.
The prism is placed 

so that the ray of light passes through the prism, entering at the
point $Q$ and emerging at the point $R$ and is picked up by a sensor
at point $S$ with coordinates $\left(-5,-6,-7\right)$. The acute
angle between $PQ$

and the normal to the top of the prism at $Q$ is $\theta$ and the
acute angle between $QR$ and the same normal is $\beta$ (sec diagram).

It is given that the top of the prism is a part of the plane $X+y+z=1$,
amd that the base of the prism is a part of the plane $x+y+z=-9$.
It is also given that the ray of light along $PQ$ is parallel to
the ray of light along $RS$ so that $P$, $Q$, $R$ and $S$ lie
in the same plane.
\begin{enumerate}
\item Find the exact coordinates of $Q$ and $R$. {[}5{]}
\item Find the values of $\cos\theta$ and $\cos\beta$. {[}3{]} 
\item Find the thickness of the prism measured in the direction of the normal
at $Q$. {[}3{]}
\end{enumerate}
Snell\textquoteleft s law states that $\sin\theta=k\sin\beta$, where
$k$ is a constant called the refractive index. 
\begin{enumerate}
\item Find $k$ for the material of this prism. {[}1{]} 
\item What can he said about the value of $k$ for a material for which
$\beta>\theta$? {[}1{]}
\end{enumerate}
