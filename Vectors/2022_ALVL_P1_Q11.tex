\item {[}ALVL/2022/P1/11{]}

A gas company has plans to install a pipeline from a gas field to
a storage facility. One part of the route for the pipeline has to
pass under a river. This part of the pipeline is in a straight line
between two points. $P$ and $Q$.

Points are defined relative to an origin $(0,0,0)$ at the gas field.
The $x$-, $y$- and $z$-axes are in the directions cast, north and
vertically upwards respectively, with units in metres. $P$ has coordinates
$\left(1136,92,p\right)$ and $Q$ has coordinates $\left(200,20,-15\right)$.
\begin{enumerate}
\item The length of the pipeline $PQ$ is 939m. Given that the level of
$P$ is below that of $Q$. find the value of $p$.\hfill{} {[}3{]}
\end{enumerate}
A thin layer of rock lies below the ground. This layer is modelled
as a plane. Three points in this plane are$\left(400,600,-20\right)$,
$\left(500,200,-70\right)$ and $\left(600,-340,-50\right)$.
\begin{enumerate}
\item[(b)]  Find the cartesian equation of this plane.\hfill{} {[}4{]}
\item[(c)] Hence find the coordinates of the point where the pipeline meets
the rock.\hfill{} {[}4{]}
\item[(d)]  Find the angle that the pipeline between the points $P$ and $Q$
makes with the horizontal.\hfill{} {[}2{]}
\end{enumerate}
