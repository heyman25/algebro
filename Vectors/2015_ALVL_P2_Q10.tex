\item {[}ALVL/2015/P2/10{]}

In an experiment the following information was gathered about air
pressure $P$, measured in inches of mercury, at different heights
above sea-level $h$, measured in feet. 
\noindent \begin{center}
\begin{tabular}{|c|c|c|c|c|c|c|c|c|c|c|}
\hline 
$h$ & 2000 & 5000 & 10 000 & 15 000 & 20 000 & 25 000 & 30 000 & 35 000 & 40 000 & 45 000\tabularnewline
\hline 
$P$ & 27.8 & 24.9 & 20.6 & 16.9 & 13.8 & 11.1 & 8.89 & 7.04 & 5.52 & 4.28\tabularnewline
\hline 
\end{tabular} 
\par\end{center}
\begin{enumerate}
\item Draw a scatter diagram for these values, labelling the axes. \hfill{}
{[}1{]}
\item Find, correct to 4 decimal places, the product moment correlation
coefficient between 
\begin{enumerate}
\item $h$ and $P$, 
\item $\ln h$ and $P$, 
\item $\sqrt{h}$ and $P$. \hfill{}{[}3{]}
\end{enumerate}
\item Using the most appropriate case from part (ii), find the equation
which best models air pressure at different heights. \hfill{}{[}3{]}
\item Given that 1 metre = 3.28 feet, re-write your equation from part (iii)
so that it can be used to estimate the air pressure when the height
is given in metres. \hfill{} {[}2{]}
\end{enumerate}
