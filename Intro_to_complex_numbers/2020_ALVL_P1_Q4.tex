\item {[}ALVL/2020/P1/4{]}

\textbf{Do not use a calculator in answering this question}. 

Three complex numbers are $z_{1}=1+\sqrt{3}i$, $z_{2}=1-i$ and $z_{3}=2\left(\cos\frac{1}{6}\pi+i\sin\frac{1}{6}\pi\right)$.
\begin{enumerate}
\item Find $\frac{z_{1}}{z_{2}z_{3}}$ in the form $r\left(\cos\theta+i\sin\theta\right)$,
where $r>0$ and $-\pi<\theta\leq\pi$. \hfill{}{[}4{]}
\end{enumerate}
A fourth complex number, $z_{4}$, is such that $\frac{z_{1}z_{4}}{z_{2}z_{3}}$
is purely imaginary and $\left|\frac{z_{1}z_{4}}{z_{2}z_{3}}\right|=1$. 
\begin{enumerate}
\item Find the possible values of $z_{4}$ in the form $r\left(\cos\theta+i\sin\theta\right)$,
where $r>0$ and $-\pi<\theta\leq\pi$.\hfill{} {[}3{]}
\end{enumerate}
